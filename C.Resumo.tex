
%\chapter*{RESUMO} %O caracter "*" oculta o t�tulo "CAPITULO" e n�o conta o cap�tulo tamb�m

%\thispagestyle{empty} % n�o numerar esta p�gina
\noindent
TAVARES, K. M., \textbf{Previs�o dos resultados do campeonato brasileiro de futebol utilizando redes neurais artificiais.} 2015. xxxxxxxxxxxxxxx f. Monografia de Conclus�o de Curso (Bacharel), Universidade Federal de Uberl�ndia, Uberl�ndia.
\vspace{1 cm}
\begin{center}
\textbf{RESUMO}\\
\end{center}
\vspace{1 cm}
Prever resultados de jogos de futebol � uma tarefa �rdua e por vezes dita imposs�vel. Os resultados s�o altamente inst�veis e abordagens estat�sticas tem sua efetividade limitada. Tendo isso em vista, o objetivo desse trabalho foi a idealiza��o de um sistema de predi��o de resultados tendo como base redes neurais artificiais, cujo uso visa eliminar dois problemas da predi��o de jogos de futebol: A limita��o dos m�todos estat�sticos, ao se utilizar reconhecimento de padr�es; e a elimina��o de resultados tendenciosos, ao se usar exclusivamente dados concretos de resultados hist�ricos. A linguagem de programa��o orientada a objetos Java foi utilizada para a montagem da rede neural artificial desse projeto, a qual consiste de um perceptron multicamadas utilizando a regra de treinamento supervisionado de Backpropagation. Dois conjuntos de dados de entrada foram considerados, um visando exprimir o hist�rico recente dos times e o outro visando exprimir sua estabilidade e const�ncia. Por fim, os resultados foram validados de forma a excluir alguns palpites menos confi�veis, garantindo assim uma taxa de acerto mais elevada nos restantes.
\\ % Cada duas barras invertidas pula uma linha
\\
\\
\\
\\
\\
\\
\\
\textit{Palavras Chave: Intelig�ncia Artificial, Redes Neurais Artificiais, Backpropagation, Futebol, Predi��o.}

\newpage
\noindent
TAVARES, K. M., \textbf{Prediction of the Brazilian soccer championship results using artificial neural networks} 2015. xxxxxxxxxxxxxxx f. Monografia de Conclus�o de Curso (Bacharel), Universidade Federal de Uberl�ndia, Uberl�ndia.
\vspace{1 cm}
\begin{center}
\textbf{ABSTRACT}\\
\end{center}
\vspace{1 cm}
The result prediction of soccer games is an arduous task and sometimes said to be impossible. The results are highly unstable and statistical approaches have their effectivity limited. With this in mind, the goal of this work was the idealization of a result prediction system based on artificial neural networks, the use of which aims to eliminate two problems of the soccer games prediction: The limitation of statistical methods, when using pattern recognition; and the elimination of biased results, when using only concrete data from historical results. The object oriented programming language Java was used to assemble the artificial neural network of this project, which consists of a multilayer perceptron using the supervised Backpropagation training rule. Two sets of input data were considered, one seeking to express the recent history of the teams and the other seeking to express their stability and constancy.  Finally, the results were validated to exclude some less reliable predictions, thus ensuring a higher success ratio on the others.
\\ % Cada duas barras invertidas pula uma linha
\\
\\
\\
\\
\\
\\
\\
\\
\textit{Keywords: Artificial Intelligence, Artificial Neural Networks, Backpropagation, Soccer, Prediction.}